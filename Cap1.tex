\chapter{Introducción}
\pagenumbering{arabic}
\setcounter{page}{1}
Hoy en día muchas personas tienen la necesidad de conocer quién y cuando alguien ha tenido acceso, ya sea a una zona de sus viviendas, a cierta información, a un servicio, etc. Por ejemplo, hay centros de trabajo en los que los trabajadores tienen turnos de noche y para poder acceder a su puesto de trabajo necesitan de una llave, ya sea porque a esas horas no hay portero, recepcionista, o simplemente porque la empresa no cuenta con ninguna persona que se encargue de abrirla, debido a esto los empleados deben turnarse para realizar esta tarea, interrumpiendo con esto su jornada laboral y perjudicando su productividad.
\vskip 0.5cm
Ocurre también que en muchas empresas es conveniente que no todos los empleados tengan acceso a ciertas zonas, por ejemplo, cuando en algunas áreas se trabaja con materiales peligrosos, documentación importante, donde se guarda dinero o material de mucho valor, debido a esto es necesario delimitar estas zonas creando permisos distintos en función de a qué áreas se pueden o no acceder, \citep{altumtec}.
\vskip 0.5cm
Debido a estos problemas que ocurren en el día a día, han dado origen a lo que hoy se conoce como \textit{Sistemas de Seguridad para el Control de Acceso}, que es la actividad principal en la seguridad física e informática. El control de acceso es posiblemente la medida de seguridad más importante que se puede establecer, dado que un control bien aplicado nos permite saber quién, cuándo y a que parte se accedió.
\vskip 0.5cm
Es por ello que los controles de acceso facilitan el trabajo a la empresa o lugar donde sea utilizado, estos controles tanto físicos como lógicos se encuentran entre los esquemas de protección imperativos en seguridad. Los controles de acceso físico aseguran que solo el personal autorizado tenga disposición a edificios, salas, documentos, etc. Por otro lado, los controles de acceso lógico protegen las computadoras, las instalaciones de red y los sistemas de información contra amenazas de acceso no autorizado. Estos dos tipos de accesos se basan esencialmente en la autentificación del usuario mediante la cual la identidad de un individuo se verifica a través de uno de los tres medios siguientes: por algo \textit{que conoce}, \textit{que tiene} o \textit{que es} (o por combinaciones de estos).
\vskip 0.5cm
Los sistemas de seguridad han ido evolucionando conforme al avance tecnológico, los primeros sistemas de identificación fueron los \textit{Sistemas de Identificación por Teclado}, donde el usuario utiliza un código o PIN (Número de Identificación Personal) para identificarse. Es económico y práctico porque no requiere identificadores físicos y hace innecesarios las llaves. Si bien este sistema es muy cómodo, presenta un problema que es, que los usuarios pueden compartir la clave fácilmente, lo que compromete la seguridad, \cite{fermax}.
\vskip 0.5cm
Ante este problema aparecieron los \textit{Sistemas de Identificación por Proximidad y Radiofrecuencia}, estos dos métodos cuentan con algo en común que es, que el usuario tiene un artículo identificador. En los \textit{Sistemas de Proximidad}, los usuarios usan tarjetas o llaveros electrónicos para identificarse, estos no necesitan de mantenimiento, aunque sí el sistema, pero su costo es bajo. Por otro lado, los \textit{Sistemas de Radiofrecuencia} son muy comunes en los garajes, a los que se accede con un pequeño mando a distancia, es un sistema cómodo, aunque el tipo de identificador es más costoso que el usado por proximidad. La seguridad que ofrecen estos sistemas es mayor en comparación a un \textit{Sistema de Identificación por Teclado}, ya que un usuario, solo puede utilizar una tarjeta o un llavero a la vez, el problema es que, si perdemos la tarjeta o llavero electrónico, el sistema quedaría inservible y tendría que reconstruirse, ocasionando costos mayores, \cite{fermax}.
\vskip 0.5cm
Debido a este nuevo problema aparecen los \textit{Sistemas de Identificación Biométrica}, la biometría está cada vez más presente en la vida cotidiana de las personas y, sobre todo, en los negocios. El control de acceso biométrico es un sistema de identificación basado en las cualidades fisiológicas del usuario, pero también puede basarse en el comportamiento del mismo, se trata de un sistema personal e intransferible, \cite{fermax}. Según \citep{hugo}, la biometría es más cómoda y segura que los sistemas tradicionales como las contraseñas, llaves o tarjetas.
\vskip 0.5cm
En \citep{cosentino} se explica que dentro de la biometría como sistema de acceso podemos diferenciar varios tipos de reconocimiento: identificación por geometría de la mano, huella digital, iris y retina y por voz.
\vskip 0.5cm
\textit{El Reconocimiento por Geometría de Mano}, identifica los parámetros dimensionales de la mano, que son únicos. Funciona muy bien, sobre todo en lugares donde no se utilizan guantes y es bastante robusta frente al vandalismo. Los motivos por el cual esta tecnología no se popularizó masivamente son estrictamente comerciales, y es que sólo había un fabricante en el mercado, lo que hizo que los fabricantes de sistemas de control de acceso no ayudaran a difundirla.
\vskip 0.5cm
\textit{El Reconocimiento por Huella Digital}, sin dudas la más popular de todas las biometrías. Es utilizado por los organismos de seguridad y gobiernos para la identificación de personas y/o sospechosos de delitos, y también en controles de acceso. No son resistentes al vandalismo y uno de los grandes inconvenientes es que la reparación del daño en el elemento sensor es bastante costoso, por lo que se recomienda utilizarlos en lugares no expuestos a estos inconvenientes. Algo similar ocurre con las inclemencias del tiempo, por lo que se deberán extremar los cuidados, si es necesario colocarlos en el exterior y sobre todo expuestos a la intemperie.
\vskip 0.5cm
\textit{El Reconocimiento Facial}, es una tecnología que año a año mejora en sus resultados y si bien actualmente existen fabricantes que ofrecen soluciones de este tipo, todavía no se consiguen equipos comerciales de bajo costo con prestaciones aceptables de identificación y velocidad. Básicamente consiste en identificar y calcular las distancias entre los diferentes accidentes faciales, reduciendo la imagen a un conjunto de coordenadas de puntos significativos.
\vskip 0.5cm
\textit{El Reconocimiento de Iris y Retina}, los equipos de este tipo de reconocimiento funcionan muy bien, tienen un costo relativamente accesible y no tienen contacto físico con el usuario, por lo que pueden ser ubicados detrás de un vidrio de manera de hacerlos resistentes al vandalismo. El inconveniente radica en que no son muy populares, por tratarse de equipos en los que la persona debe mirar adentro y si bien su ojo no tiene contacto con ningún elemento, de todas formas, suelen generar el rechazo de los usuarios.
\vskip 0.5cm
Y por último el \textit{Reconocimiento de Voz}, está tecnología se encuentra en una etapa similar al reconocimiento facial y probablemente algún día sean una alternativa válida.
\vskip 0.5cm
Como podemos observar, las tecnologías biométricas pueden ser una alternativa o un complemento de las técnicas de identificación y autenticación ya existentes. Debido a esto y a la poca producción de sistemas reconocedores de voz, está investigación se basará en este tipo de sistema.
\vskip 0.5cm
Con esto, finalmente podemos decir que la tecnología puede ser una herramienta sumamente útil y poderosa para reducir el riesgo a ser víctimas de intrusos en nuestras viviendas. Es por ello que nos proponemos a diseñar un sistema tecnológico que ayudará a solucionar este problema.

\section{Justificación de la investigación}
Debido a la inseguridad actual que se presenta en la ciudad de Trujillo se realizará un sistema de seguridad para una vivienda por medio de la voz, con la finalidad de proponer a los habitantes de la ciudad un sistema capaz de resguardar su vivienda.
\vskip 0.5cm
Por otro lado, cada vez hay personas menos preparadas en la manipulación de sistemas tecnológicos, debido a la poca familiaridad que tienen con estos, por ello esta investigación propone a los habitantes de la ciudad de Trujillo un sistema simple y fácil de usar.
\vskip 0.5cm
Si bien existen muchos tipos de sistemas de seguridad, donde los biométricos destacan, estos suelen tener un costo elevado para su implementación, debido a que los componentes que se utilizan para la captura y el procesamiento de los datos son caros, siendo así difícil la venta y compra de estos, es por ello que este proyecto servirá como una alternativa ante este problema, permitiendo a la sociedad la adquisición de estos sistemas a un precio cómodo.
\vskip 0.5cm
Además, actualmente no existe ninguna API (Interfaz de Programación de Aplicaciones) de reconocimiento de voz que este de manera gratuita para los desarrolladores de programas informáticos que deseen usar el reconocimiento de voz en sus aplicaciones. Si bien Google ofrece gratuitamente su API de reconocimiento de voz a los desarrolladores de aplicaciones para dispositivos Android, ocurre que para el uso de está API es necesario la conexión a internet, lo que hace que su uso tenga esta limitante, es por ello que nuestro proyecto será una alternativa para aquellos desarrolladores que no cuentan con dinero y acceso a internet, puedan usar esta tecnología en sus programas.

\section{Formulación del problema}
\begin{center} 
	?`Cómo controlar el acceso a una vivienda por reconocimiento de voz del locutor dependiente del texto?
\end{center}

\section{Hipótesis}
El acceso a una vivienda puede ser controlado por reconocimiento de voz del locutor dependiente del texto haciendo uso de los métodos de cuantificación de la señal de voz MFCC Y DTW.

\section{Objetivos}
\subsection{Generales}
Desarrollar un sistema de seguridad para el control de acceso a una vivienda que funcione en base al reconocimiento de voz dependiente del texo.
\subsection{Específicos}
\begin{enumerate}
\item[a)]Aplicar la metodología propuesta por Tammy Noergaard para el diseño de arquitecturas para sistemas empotrados o embebidos.
\item[b)]Analizar la transmisión, aplicación y manipulación de las señales de voz para poderlas utilizar en el desarrollo del sistema de seguridad de control de acceso.
\item[c)]Implementar un algoritmo para la extracción de caracteristicas de la señal de voz para la etapa de entrenamiento.
\item[d)]Implementar un algoritmo para la comparación entre dos señales de voz para la etapa de reconocimiento.
\item[e)]Implementar un algoritmo para la eliminación de ruido externo o ambiental producida por alguna fuente de ruido.
\item[f)]Implementar un algoritmo para la detección de inicio y fin de la señal de voz.
\item[g)]Integrar y evaluar los algoritmos de los módulos de preprocesamiento, entrenamiento y reconocimiento, de manera que se genere un modelo con menor tasa de error en el reconocimiento de voz.
\item[h)]Desarrollar un software que funcione como interfaz con el usuario y que además realice las comparaciones y cálculos necesarios para el reconocimiento del locutor.
\item[i)]Construir un prototipo hardware para el acceso a una vivienda.
\item[j)]Implementar un canal de comunicación inalámbrica entre el hardware y el software.
\item[k)]Evaluar la seguridad de nuestro sistema ante intrusos en un ambiente real.
\item[l)]Evaluar el rendimiento y minimizar el tiempo de respuesta para el reconocimiento de voz.
\end{enumerate}

\section{Estructura de la tesis}
El presente trabajo está dividido en cinco capítulos. El primer capítulo presenta los aspectos generales de la investigación realizada tal como justificación, formulación del problema, la hipótesis, los objetivos y la estructura de la tesis.
\vskip 0.5cm
En el segundo capítulo se presenta el marco referencial teórico y soporte del tema, contemplando los conceptos básicos que se debe de tener en cuenta para el desarrollo de un sistema de seguridad para el control de acceso por reconocimiento de voz, así como algunas consideraciones con respecto al habla. Además, se dará a conocer los métodos y las etapas por las cuales debe pasar un reconocedor de voz para poder procesar y manejar las señales de voz por medio de un software. Finalmente el método empleado en la investigación.
\vskip 0.5cm
El tercer capítulo muestra los procesos para la construcción de nuestro sistema de seguridad para el control de acceso, dividido en tres etapas: la primera la construcción del algoritmo de reconocimiento de voz y su integración en una aplicación Android, la segunda la construcción del prototipo hardware y la tercera el canal de comunicación entre los dos módulos.
\vskip 0.5cm
El cuarto capítulo se muestran los resultados y discusión obtenida en la investigación. El quinto capítulo contiene las consideraciones finales obtenidas en esta tesis. Primero se presentan las conclusiones, seguida de las recomendaciones para futuras investigaciones relacionadas al tema en cuestión. 
\vskip 0.5cm
Por último, las referencias bibliográficas que se usaron para la investigación en esta tesis y en apéndice se presenta el instrumento que se usará para constratar la hipótesis. Finalmente la declaración jurada y la autorización de la tesis.