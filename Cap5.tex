\chapter{Consideraciones finales}
En esta tesis se desarrolló un sistema de seguridad por reconocimiento de voz en una plataforma de simulación, basado en la extracción de características de una palabra hablada. Se emplearon los algoritmos MFCC para construir los patrones de voz y DTW para la comparación entre ellos. El sistema es dependiente del texto y se hicieron experimentos para un sistema de conjunto cerrado y para un sistema de conjunto abierto. Se hicieron pruebas variando los valores de los parámetros de los algoritmos empleados con el fin de generar el porcentaje más alto de tasa de acierto.

\section{Conclusiones}
\begin{enumerate}
\item[1.]Debido a que un sistema de seguridad biométrico se divide en hardware y software, el uso de la metodología propuesta por Tammy Noergaard en \citep{tammy}, para el diseño de sistemas embebidos permitió seguir un orden específico entre ellos al momento de implementar los diferentes módulos necesarios para el software y hardware.

\item[2.]Se observó que el sistema funciona mejor con una frecuencia de muestreo de 16000 Hz en las grabaciones, si bien en la mayoría de reconocedores del habla se emplea una frecuencia de 8000 Hz, en nuestro caso no fue así, esto porque aparte de reconocer la palabra es necesario reconocer al locutor, por lo que se requiere más información de la señal de voz. Además de esta configuración se tuvo las siguientes características de grabación: un solo canal (mono), una resolución 16 bits y un ordenamiento de bits Little Endian.

\item[3.]En la etapa de entrenamiento se empleó el algoritmo MFCC con el que se obtuvo una mayor tasa de acierto en el reconocimiento de voz, en lugar de los MFCC delta y doble delta, ver Tablas del \ref{table:tabla3.3} al \ref{table:tabla3.9}.

\item[4.]En la etapa de reconocimiento se empleó el algoritmo DTW simétrico en lugar del asimétrico, esto porque de la forma simétrica se obtiene una mayor información de la matriz de distancias permitiendo escoger la distancia con menor peso, ver Tablas del \ref{table:tabla3.13} al \ref{table:tabla3.16}.

\item[5.]Para la eliminación del ruido se empleó el algoritmo NLMS, esto porque este usa un tamaño de paso de adaptación dinámico lo que permite que la señal se filtre de una mejor manera sin distorsionarla, sin embargo, este algoritmo tiene una limitación y es que funciona correctamente en ambientes con un ruido no máximo de 80 dBC, ver Figuras \ref{fig:figura3.80} y \ref{fig:figura3.83}.

\item[6.]Para la detección de inicio y fin de la señal de voz se usó el algoritmo de Rabiner \& Sambur propuesto en \citep{unam}, esto porque el algoritmo definido en \citep{rabiner} se requiere que la frecuencia de muestreo sea de 10 KHz para su correcto funcionamiento, sin embargo como se dijo anteriormente se optó por una frecuencia de muestreo de 16 KHz, es por esto que se obtuvo mejores resultados con el primer algoritmo, ver Figuras \ref{fig:figura3.75} y \ref{fig:figura3.76}.

\item[7.]Luego de experimentar el uso de diversos algoritmos así como la variación de los parámetros de sus funciones (Sección 3.3), podemos decir que la seguridad por voz irá en aumento al paso de las diferentes etapas del sistema como filtrado, detección de inicio y fin de la señal de voz, normalización, obtención de parámetros, etc. tecnologías que vallan mejorando la confiabilidad y velocidad en la toma de decisión, estas a su vez ayudarán en la implementación de sistemas de seguridad robustos y eficientes.

\item[8.]En un sistema de seguridad por reconocimiento de voz, la toma de decisión ante un reconocimiento de un locutor, puede ser dificultoso debido a varios factores, como el nivel de ruido donde se hacen las grabaciones, la similitud de las palabras y los diferentes estados de ánimo de la persona, por ejemplo, si el usuario dice la palabra en un tono de extrema alegría o trizteza, o si está en un estado alcohólico o inconveniente, esto modificará las características de su voz, provocando una disminución en la probabilidad de reconocerlo.
\vskip 0.5cm
Para corregir este problema se podría proponer tener más bases de datos del mismo usuario, pero en diferentes estados de ánimo, así el usuario antes de identificarse podría indicar su estado de ánimo y el programa relacionaría su grabación en el momento, pero esto podría facilitar en gran medida el acceso a otros usuarios es por esa razón por la cual no se implementó esta solución, y se optó por que el usuario grabara de forma correcta.

\item[9.]Al utilizar umbrales ajustados se obtuvo un porcentaje de rechazo de intrusos del 100\%, ver Tablas \ref{table:tabla4.13} y \ref{table:tabla4.14}. Sin embargo, existe un compromiso entre seguridad y la tasa de identificación, es decir, entre más seguro sea el sistema a intrusos, más rechazos de usuarios válidos se tendrán y por lo tanto la tasa de identificación disminuíra, ver Tablas \ref{table:tabla4.1} y \ref{table:tabla4.2}.

\item[10.]Todo sistema biométrico o sistema de seguridad para el control de acceso su tiempo de respuesta no debe de ser mayor a 5 segundos, sin embargo observamos que para nuestro sistema con una cantidad mayor a 60 patrones característicos (patron MFCC) esto ya no se cumple ver Figuras \ref{fig:figura4.5} y \ref{fig:figura4.6}, debido a que el sistema toma un tiempo para poder reajustar los umbrales de decisión, además al aumentar el número de usuarios en el sistema, la similitud entre cada uno de ellos será mayor, causando que sus coeficientes MFCC podrían converger en un punto provocando una tasa de acierto menor en el reconocimiento, para evitar esto se tomó umbrales con valores estáticos $U1=0.51$ y $U2=0.31$.

\item[11.]Se implementó el algoritmo de construcción de patron de referencia, que consiste en obtener un patron característico a partir de un conjunto de patrones de un usuario, reduciendo así el número de comparaciones que realiza el sistema para reconocer la voz del locutor, sin embargo no se empleo esta técnica en nuestro sistema debido a que se obtuvo una tasa de error del 8\% en el reconocimiento de los usuarios del sistema, ver Tabla \ref{table:tabla4.34}.

\item[12.]Nuestro algoritmo contribuye en el reconocimiento de voz del locutor con una tasa de acierto del 95\% para el caso de un sistema de conjunto cerrado y una tasa de acierto del 100\% ante intrusos en el caso de sistema de conjunto abierto, mucho mejor que los resultados de 79.63\% y 92.6\% para sistemas abierto y cerrado respectivamente obtenido por \cite{eyra} y mucho más proximo y mejor al resultado de 94.44\% y 98.06\% obtenido en la investigación hecha por \cite{varela}.

\item[13.]En conclusión, el objetivo general que trata sobre implementar un sistema de seguridad para el control de acceso a una vivienda que funcione en base al reconocimiento de voz que sea dependiente del texto fue conseguido a través del algoritmo conformado por los métodos MFCC Y DTW.
\end{enumerate}

\section{Trabajos futuros}
\begin{enumerate}
\item[1.]Se aprendió que para realizar un reconocedor de voz, se necesitan muchas partes tanto software como hardware que realicen cada una de sus funciones de manera adecuada, siendo el software el de mayor importancia, gracias a este se pueden programar los algoritmos para poder reconocer al locutor, pero el hardware también es importante debido a que al usar un micrófono mejor, se puede obtener una mejor calidad en la señal capturada y por lo tanto, dejar que el software se encargue sólo del procesamiento de la señal y no tenga que desperdiciar tiempo en limpiarla, filtrarla y normalizarla.

\item[2.]El ruido externo puede ser un gran problema para el sistema al momento de realizar el reconocimiento de voz, debido a que el algoritmo que se usó funciona correctamente con niveles de ruido menores a 80 dBC lo que resulta una limitante, ante este problema una alternativa como se dijo anteriormente es en usar un mejor micrófono, pero puede que su costo sea muy elevado, por lo que se recomienda estudiar y evaluar los otros algoritmos adaptativos de eliminación de ruido que vimos en la teoría, ver Tabla \ref{table:tabla24}.

\item[3.]Al igual que el ruido externo el eco es otro problema que ocasiona que el reconocimiento de voz no se realice correctamente, por lo que es necesario implementar un algoritmo de eliminación de eco o usar un micrófono que filtre el eco.

\item[4.]Tal como se dijo el hardware de un sistema de reconocimiento de voz es muy importante para su correcto funcionamiento, y es que como vimos en las pruebas que se realizaron en un dispositivo móvil con un procesador de 4x1.8 GHz, el tiempo de ejecución presentó una limitante y es que a partir de una base de datos mayor a 60 patrones característicos el tiempo de respuesta es mayor a 5 segundos, debido al gran número de usuarios o al número de patrones que cada uno de estos tienen, incumpliendo así uno de los requisitos principales de un sistema biométrico de tiempo real. Por lo que se recomienda implementar el algoritmo de reconocimiento de voz en un servidor, esto porque una computadora tiene mayor velocidad de procesamiento que el de un dispositivo móvil, reduciendo así el tiempo de respuesta del sistema.

\item[5.]El sistema no soporta el reconocimiento de voz continuo solo palabras aisladas, es decir el usuario solo puede pronunciar una palabra para el reconocimiento de voz y no un conjunto de ellas (frase), por lo que se podría implementar un algoritmo para este caso. Existe una variante del algoritmo de Rabiner \& Sambur el cual usa la detección de pulsos de energía, por lo que se recomienda estudiarla.

\item[6.]En este sistema se pueden producir tasas de aceptación altas si todos los usuarios son muy cooperativos, es decir, si la forma de pronunciación del locutor es muy parecida a las palabras habladas, sin embargo, en un sistema real esto no sucede. Por lo tanto, se recomienda hacer un estudio de los algoritmos basados en Redes Neuronales, el uso de probabilidades como son los Modelos Ocultos de Markov, Redes Bayesianas, etc. y cómo aplicarlos al reconocimiento de voz del locutor.

\item[7.]Puede que existan usuarios o intrusos con las voces similares, esto podría ser un problema para nuestro sistema de seguridad por reconocimiento de voz, es por ello que se recomienda integrar a nuestro sistema otro tipo de sistema biométrico como, por ejemplo, el reconocimiento facial, donde podríamos usar la cámara del dispositivo móvil.
\end{enumerate}
