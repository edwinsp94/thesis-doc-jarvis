\chapter{Consideraciones finales}
En esta tesis se desarrolló un sistema de seguridad por reconocimiento de voz en una plataforma de simulación, basado en la extracción de características de una palabra hablada. Se emplearon los algoritmos MFCC para construir los patrones de voz y DTW para la comparación entre ellos. El sistema es dependiente del texto y se hicieron experimentos para un sistema de conjunto cerrado y para un sistema de conjunto abierto. Se hicieron pruebas variando los parámetros de los algoritmos empleados con el fin de obtener los parámetros óptimos con los cuales se genera el porcentaje más alto en la tasa de acierto.

\section{Conclusiones}
\begin{enumerate}
\item[1.]En un sistema de reconocimiento de voz, la toma de decisión es dificultoso debido a varios factores como el nivel de ruido donde se hacen las grabaciones, la similitud de las palabras y los estados de ánimo de la persona.

\item[2.]Un sistema de seguridad por voz será más confiable ya que pasará por diversos métodos siendo los más importantes los Coeficientes Cepstrales en la Frecuencia Mel (MFCC), el Alineamiento Temporal Dinámico (DTW) y las condiciones implementadas por el programador, disminuyendo la probabilidad de error al reconocer al locutor. 
\vskip 0.5cm
Aunque un aspecto importante a considerar y que el programador no puede corregir es si el usuario dice la palabra en un tono de extrema alegría o tristeza ya que esto modifica las características de su voz afectando los coeficientes MFCC, como a su vez lo haría si el usuario está en estado alcohólico o inconveniente. 
\vskip 0.5cm
Para corregir este problema se propuso tener más bases de datos del mismo usuario, pero en diferentes estados de ánimo, así el usuario antes de identificarse podría indicar su estado de ánimo y el programa relacionaría su grabación en el momento, pero esto podría facilitar en gran medida el acceso a otros usuarios es por esa razón por la cual que no se implementó esta solución, y se optó por que el usuario grabara de forma correcta.

\item[3.]La seguridad por voz irá en aumento al paso de las diferentes etapas del sistema como filtrado, detección de inicio y fin de la señal de voz, normalización, obtención de parámetros, tecnologías que vallan mejorando la confiabilidad y velocidad en la toma de decisión, estas a su vez ayudarán en la implementación de sistemas de seguridad eficientes.

\item[4.]El sistema de reconocimiento de voz incrementará en gran medida la seguridad en la vivienda y será la base de futuras implementaciones de seguridad total automatizada para una vivienda, incluyendo el control de ventanas, luces, puertas, etc. Haciendo de este un proyecto base de Domótica.

\item[5.]Debido a que un sistema empotrado se divide en hardware y software, el uso de la metodología propuesta por Tammy Noergaard en \citep{tammy}, para el diseño de sistemas empotrados permitió seguir un orden específico entre ellos al momento de implementar los diferentes módulos necesarios para los bloques de software y hardware, respectivamente.

\item[6.]Se observó que el sistema funciona mejor con una frecuencia de muestreo de 16000 Hz en las grabaciones, si bien en la mayoría de reconocedores del habla se emplea una frecuencia de 8000 Hz, en nuestro caso no será así, esto porque aparte de reconocer la palabra es necesario reconocer al locutor, por lo que se requiere más información de la señal.

\item[7.]En la eliminación del ruido se empleó el algoritmo NLMS en vez del LMS, esto porque el NLMS usa un tamaño de paso de adaptación dinámico lo que permite que la señal se filtre de una mejor manera sin distorsionarla, sin embargo, este algoritmo tiene una limitación y es que funciona correctamente en ambientes con un ruido no máximo de 80 dBC.
\newpage
\item[8.]Para la detección de inicio y fin de la señal de voz se usó el algoritmo de Rabiner \& Sambur (Tipo 1) propuesto en \citep{unam}, esto porque el algoritmo de Rabiner \& Sambur (Tipo 2) definido en \citep{rabiner}, se requiere que la frecuencia de muestreo sea de 10000 Hz para su correcto funcionamiento y como se dijo en el punto 7 para nuestro sistema se optó por una frecuencia de muestreo de 16000 Hz, es por eso que se obtuvo mejores resultados con el primer algoritmo, ver Figuras \ref{fig:figura3.75} y \ref{fig:figura3.76}.

\item[9.]En la etapa de entrenamiento se empleó el algoritmo MFCC con el que se obtuvo una mayor tasa de acierto en el sistema, en vez de los MFCC delta y MFCC doble delta, ver Tablas del \ref{table:tabla3.2} al \ref{table:tabla3.9}.

\item[10.]En la etapa de reconocimiento se empleó el algoritmo DTW simétrico en vez del asimétrico, esto porque de la forma simétrica se obtiene una mayor información del entorno para escoger la distancia más corta, ver Tablas del \ref{table:tabla3.13} al \ref{table:tabla3.16}.

\item[11.]Al aumentar el número de locutores en el sistema la base de datos sería más extensa, y la similitud entre cada usuario y sus coeficientes MFCC podrían converger en un punto causando un porcentaje de aceptación menor.

\item[12.]Existe un compromiso entre seguridad y el porcentaje de identificación, es decir, entre más seguro sea el sistema a intrusos, más rechazos de usuarios válidos se tendrán y por lo tanto el porcentaje de identificación se reducirá, ver Tablas \ref{table:tabla4.1} y \ref{table:tabla4.2}.

\item[13.]Al utilizar umbrales ajustados se obtiene un porcentaje de rechazo de intrusos del 100\%, ver Tablas \ref{table:tabla4.13} y \ref{table:tabla4.14}.
\end{enumerate}

\section{Trabajos futuros}
\begin{enumerate}
\item[1.]Se aprendió que para realizar un reconocedor de voz, se necesitan muchas partes tanto software como hardware que realicen cada una de sus funciones de manera adecuada, siendo el software el de mayor importancia, gracias a este se pueden programar los algoritmos para poder reconocer al locutor, pero el hardware también tiene su importancia debido a que al usar un micrófono mejor, se puede obtener una mejor calidad en la señal adquirida y por lo tanto, dejar que el software se encargue sólo del procesamiento de la señal y no tenga que desperdiciar tiempo en limpiarla, filtrarla y normalizarla.

\item[2.]El ruido externo puede ser un gran problema para el sistema al momento de realizar el reconocimiento de voz, debido que el algoritmo que se usó funciona correctamente con niveles de ruido menores a 80 dBC lo que resulta una limitante, ante este problema una alternativa como se dijo anteriormente es en usar un mejor micrófono, pero puede que su costo sea muy elevado, por lo que se recomienda estudiar y evaluar los otros algoritmos adaptativos de eliminación de ruido que vimos en la teoría, ver Tabla \ref{table:tabla24}.

\item[3.]Lo mismo sucede con el eco, este es otro problema que ocasiona que el reconocimiento de voz no se realice correctamente, por lo que es necesario implementar un algoritmo de eliminación de eco o usar un micrófono que filtren el eco.

\item[4.]Tal como se explicó anteriormente el hardware del sistema es muy importante para un correcto funcionamiento de un sistema de reconocimiento de voz, y es que tal como vimos en las pruebas las cuales se realizaron en un dispositivo móvil con un procesador de 1.8 GHz de 4 núcleos, el tiempo de ejecución presentó una limitante y es que a partir de una base de datos mayor a 60 audios el tiempo de respuesta es mayor de 5 segundos, debido al gran número de usuarios o al número de patrones que este tiene, incumpliendo así uno de los requisitos principales de un sistema biométrico de tiempo real. 
\vskip 0.5cm
Por lo que se recomienda implementar el algoritmo de reconocimiento de voz en un servidor, esto es porque una computadora tiene un mejor procesador que el de un dispositivo móvil, a mayor velocidad en el procesador menor será el tiempo de ejecución.

\item[5.]El sistema no soporta el reconocimiento de voz continuo solo palabras aisladas, es decir el usuario solo puede pronunciar una palabra para el reconocimiento de voz y no un conjunto de ellas (frase), por lo que se podría implementar un algoritmo para este caso. Existe una variante del algoritmo de Rabiner \& Sambur el cual usa la detección de pulsos de energía, por lo que se recomienda estudiarla.

\item[6.]En este sistema se pueden producir porcentajes de aceptación altos si todos los locutores son muy cooperativos, es decir, si la forma de pronunciación del locutor es muy parecida para las palabras habladas, sin embargo, en un sistema real esto no sucede, por lo general los locutores son no cooperativos y pronuncian la palabra en forma diferente haciendo que el porcentaje de aceptación disminuya. Por lo tanto, se recomienda hacer un estudio de los algoritmos basados en Redes Neuronales, el uso de probabilidades como son los Modelos Ocultos de Markov, Redes Bayesianas, etc. y cómo aplicarlos al reconocimiento de locutor.

\item[7.]Puede que existan usuarios o intrusos con las voces similares, esto podría ser un problema para nuestro sistema de seguridad por reconocimiento de voz, es por ello que se recomienda integrar a nuestro sistema otro tipo de sistema biométrico como, por ejemplo, el reconocimiento facial, donde podemos usar la cámara del dispositivo móvil.
\end{enumerate}

