\subsection{Hallar BC y M}
En esta prueba hallaremos $BC$ y $M$, parámetros del algoritmo MFCC empleado para obtener el patrón característico de una señal de voz, teniendo la siguiente configuración del algoritmo:\\
- Filtro Preénfasis \\
\hspace*{1cm} $\alpha$ = Desde 0.80 hasta 0.97 \\
- Segmentación por Hamming\\
\hspace*{1cm} V = 30 ; \qquad S = 12, 14, 15 y 16\\
- MFCC\\
\hspace*{1cm} BC = 13, 15, 18, 20, 22 y 24 ; \qquad M = 13, 15, 26 y 39\\
- DTW\\
\hspace*{1cm} P = 1 Simétrico ; \qquad R = 40 ; \qquad D = Euclidiana Cuadrática

\vskip 0.5cm
A continuación, veremos algunos de los resultados (mejores) que se obtuvieron para los MFCC, los Delta y Doble Delta.

\begin{center}
\begin{table}[H]
\centering
\caption{\small{Resultados del testeo del sistema para BC = 20, 22 y 24 con M = 13.}}
\label{table:tabla3.3}
\vskip 0.2cm
\scalebox{0.54}{
\begin{tabular}{|c|c|c|c|c|c|c|c|c|c|c|c|c|c|}
\hline 
{} & 
\multicolumn{4}{c|}{BC=20 M=13 V=30 P=S1 R=40} & 
\multicolumn{4}{c|}{BC=22 M=13 V=30 P=S1 R=40} & 
\multicolumn{4}{c|}{BC=24 M=13 V=30 P=S1 R=40}  \\ 
\hline 
$\alpha$ & 
S=16 & S=15 & S=14 & S=12 & 
S=16 & S=15 & S=14 & S=12 & 
S=16 & S=15 & S=14 & S=12 \\ 
\hline 
0.8 & 
5 & 7 & 6 & 6 & 4 & 4 & 3 & 3 & 5 & 4 & 3 & 3 \\
\hline 
0.81 & 
5 & 7 & 7 & 6 & 4 & 4 & 3 & 3 & 5 & 4 & 3 & 3 \\
\hline 
0.82 & 
5 & 7 & 8 & 7 & 4 & 4 & 3 & 3 & 5 & 4 & 3 & 3 \\
\hline 
0.83 & 
7 & 6 & 5 & 7 & 4 & 4 & 3 & 3 & 5 & 4 & 3 & 3 \\
\hline 
0.84 & 
6 & 7 & 7 & 7 & 4 & 4 & 3 & 3 & 5 & 4 & 3 & 3 \\
\hline 
0.85 & 
6 & 6 & 8 & 7 & 4 & 4 & 3 & 3 & 5 & 4 & 3 & 3 \\
\hline 
0.86 & 
6 & 6 & 7 & 7 & 4 & 4 & 3 & 3 & 5 & 4 & 3 & 3 \\
\hline 
0.87 & 
5 & 6 & 7 & 6 & 4 & 4 & 3 & 3 & 5 & 4 & 3 & 3 \\
\hline 
0.88 & 
6 & 7 & 6 & 6 & 4 & 4 & 3 & 3 & 5 & 4 & 3 & 3 \\
\hline 
0.89 & 
6 & 7 & 8 & 8 & 4 & 4 & 3 & 3 & 5 & 4 & 3 & 3 \\
\hline 
0.9 & 
8 & 6 & 5 & 6 & 4 & 4 & 3 & 3 & 5 & 4 & 3 & 3 \\
\hline 
0.91 & 
8 & 6 & 5 & 6 & 4 & 4 & 3 & 3 & 5 & 4 & 3 & 3 \\
\hline 
0.92 & 
7 & 4 & 4 & 5 & 4 & 4 & 3 & 3 & 5 & 4 & 3 & 3 \\
\hline 
0.93 & 
6 & 5 & 6 & 6 & 4 & 4 & 3 & 3 & 5 & 4 & 3 & 3 \\
\hline 
0.94 & 
7 & 6 & 5 & 6 & 4 & 4 & 3 & 3 & 5 & 4 & 3 & 3 \\
\hline 
0.95 & 
7 & 5 & 6 & 5 & 4 & 4 & 3 & 3 & 4 & 4 & 3 & 3 \\
\hline 
0.96 & 
5 & 4 & 6 & 5 & 4 & 4 & 3 & 3 & 4 & 4 & 3 & 3 \\
\hline 
0.97 & 
3 & 4 & 5 & 6 & 4 & 4 & 3 & 3 & 4 & 4 & 3 & 3 \\
\hline 
min & 
3 & 4 & 4 & 2 & 4 & 4 & 3 & 3 & 4 & 4 & 3 & 3 \\
\hline 
prom & 
6 & 5.888889 & 6.166667 & 6.222222 & 4 & 4 & 3 & 3 & 4.833333 & 4 & 3 & 3 \\
\hline 
error & 
\multicolumn{4}{c|}{6.069444444} & 
\multicolumn{4}{c|}{3.5} & 
\multicolumn{4}{c|}{3.708333333}\\ 
\hline 
\end{tabular} 
}
\begin{center}
\vskip 0.2cm
{\small{Fuente: Elaboración propia}}
\end{center}
\end{table}
\end{center}

\vskip -0.5cm

\begin{center}
\begin{table}[H]
\centering
\caption{\small{Resultados del testeo del sistema para BC = 20, 22 y 24 con M = 15.}}
\label{table:tabla3.5}
\vskip 0.2cm
\scalebox{0.54}{
\begin{tabular}{|c|c|c|c|c|c|c|c|c|c|c|c|c|c|}
\hline 
{} & 
\multicolumn{4}{c|}{BC=20 M=15 V=30 P=S1 R=40} & 
\multicolumn{4}{c|}{BC=22 M=15 V=30 P=S1 R=40} & 
\multicolumn{4}{c|}{BC=24 M=15 V=30 P=S1 R=40}  \\ 
\hline 
$\alpha$ & 
S=16 & S=15 & S=14 & S=12 & 
S=16 & S=15 & S=14 & S=12 & 
S=16 & S=15 & S=14 & S=12\\ 
\hline 
0.8 & 
6 & 6 & 7 & 6 & 5 & 4 & 3 & 3 & 3 & 4 & 3 & 3 \\
\hline 
0.81 & 
6 & 6 & 6 & 6 & 5 & 4 & 3 & 3 & 3 & 4 & 3 & 3 \\
\hline 
0.82 & 
7 & 7 & 6 & 7 & 5 & 4 & 3 & 3 & 3 & 4 & 3 & 3 \\
\hline 
0.83 & 
7 & 6 & 6 & 7 & 5 & 4 & 3 & 3 & 3 & 4 & 3 & 3 \\
\hline 
0.84 & 
6 & 6 & 6 & 7 & 5 & 4 & 3 & 3 & 3 & 4 & 3 & 3 \\
\hline 
0.85 & 
6 & 6 & 7 & 7 & 5 & 4 & 3 & 3 & 3 & 4 & 3 & 3 \\
\hline 
0.86 & 
6 & 6 & 8 & 7 & 5 & 4 & 3 & 3 & 3 & 4 & 3 & 3 \\
\hline 
0.87 & 
5 & 6 & 7 & 7 & 4 & 4 & 3 & 3 & 3 & 4 & 3 & 3 \\
\hline 
0.88 & 
6 & 7 & 7 & 7 & 4 & 4 & 3 & 3 & 3 & 4 & 3 & 3 \\
\hline 
0.89 & 
6 & 6 & 7 & 7 & 4 & 4 & 3 & 3 & 3 & 4 & 3 & 3 \\
\hline 
0.9 & 
7 & 6 & 6 & 7 & 4 & 4 & 3 & 3 & 4 & 4 & 3 & 3 \\
\hline 
0.91 & 
8 & 6 & 4 & 6 & 4 & 4 & 3 & 3 & 4 & 4 & 3 & 3 \\
\hline 
0.92 & 
7 & 6 & 5 & 6 & 4 & 4 & 3 & 3 & 4 & 4 & 3 & 3 \\
\hline 
0.93 & 
6 & 4 & 6 & 6 & 4 & 4 & 3 & 3 & 4 & 4 & 3 & 3 \\
\hline 
0.94 & 
7 & 6 & 6 & 5 & 4 & 3 & 3 & 3 & 4 & 4 & 3 & 3 \\
\hline 
0.95 & 
6 & 5 & 5 & 6 & 4 & 3 & 3 & 3 & 3 & 4 & 3 & 3 \\
\hline 
0.96 & 
5 & 4 & 6 & 5 & 4 & 3 & 3 & 3 & 3 & 4 & 3 & 3 \\
\hline 
0.97 & 
4 & 4 & 5 & 5 & 4 & 3 & 3 & 3 & 3 & 3 & 3 & 3 \\
\hline 
min & 
4 & 4 & 4 & 5 & 4 & 3 & 3 & 3 & 3 & 3 & 3 & 3 \\
\hline 
prom & 
6.166667 & 5.722222 & 6.111111 & 6.333333 & 4.3888889 & 3.777778 & 3 & 3 & 3.2777778 & 3.944444 & 3 & 3 \\
\hline 
error & 
\multicolumn{4}{c|}{6.083333333} & 
\multicolumn{4}{c|}{3.541666667} & 
\multicolumn{4}{c|}{3.305555556}\\ 
\hline 
\end{tabular} 
}
\begin{center}
\vskip 0.2cm
{\small{Fuente: Elaboración propia}}
\end{center}
\end{table}
\end{center}

\vskip -0.5cm

\begin{center}
\begin{table}[H]
\centering
\caption{\small{Resultados del testeo del sistema para BC = 20, 22 y 24 con M = 26.}}
\label{table:tabla3.7}
\vskip 0.2cm
\scalebox{0.54}{
\begin{tabular}{|c|c|c|c|c|c|c|c|c|c|c|c|c|c|}
\hline 
{} & 
\multicolumn{4}{c|}{BC=20 M=26 V=30 P=S1 R=40} & 
\multicolumn{4}{c|}{BC=22 M=26 V=30 P=S1 R=40} & 
\multicolumn{4}{c|}{BC=24 M=26 V=30 P=S1 R=40}  \\ 
\hline 
$\alpha$ & 
S=16 & S=15 & S=14 & S=12 & 
S=16 & S=15 & S=14 & S=12 & 
S=16 & S=15 & S=14 & S=12\\ 
\hline 
0.8 & 
15 & 12 & 10 & 9 & 5 & 5 & 4 & 4 & 4 & 5 & 4 & 4 \\
\hline 
0.81 & 
12 & 13 & 11 & 11 & 5 & 5 & 4 & 4 & 4 & 5 & 4 & 4 \\
\hline 
0.82 & 
14 & 11 & 13 & 10 & 5 & 5 & 4 & 4 & 4 & 5 & 4 & 4 \\
\hline 
0.83 & 
15 & 11 & 11 & 9 & 5 & 5 & 4 & 4 & 4 & 5 & 4 & 4 \\
\hline 
0.84 & 
13 & 10 & 11 & 9 & 5 & 5 & 4 & 4 & 4 & 5 & 4 & 4 \\
\hline 
0.85 & 
14 & 10 & 11 & 10 & 5 & 5 & 4 & 4 & 4 & 5 & 4 & 4 \\
\hline 
0.86 & 
13 & 9 & 10 & 10 & 5 & 5 & 4 & 4 & 4 & 5 & 4 & 4 \\
\hline 
0.87 & 
12 & 9 & 11 & 12 & 5 & 5 & 4 & 4 & 4 & 5 & 4 & 4 \\
\hline 
0.88 & 
12 & 11 & 8 & 12 & 5 & 5 & 4 & 4 & 4 & 5 & 4 & 4 \\
\hline 
0.89 & 
12 & 10 & 9 & 12 & 5 & 5 & 4 & 4 & 4 & 5 & 4 & 4 \\
\hline 
0.9 & 
12 & 7 & 11 & 9 & 5 & 5 & 4 & 4 & 4 & 5 & 4 & 4 \\
\hline 
0.91 & 
14 & 9 & 12 & 8 & 5 & 5 & 4 & 4 & 4 & 5 & 4 & 4 \\
\hline 
0.92 & 
12 & 9 & 12 & 9 & 5 & 5 & 4 & 4 & 4 & 5 & 4 & 4 \\
\hline 
0.93 & 
13 & 9 & 10 & 9 & 5 & 5 & 4 & 4 & 4 & 5 & 4 & 4 \\
\hline 
0.94 & 
12 & 12 & 8 & 9 & 5 & 5 & 4 & 4 & 4 & 5 & 4 & 4 \\
\hline 
0.95 & 
13 & 11 & 7 & 10 & 5 & 5 & 4 & 4 & 4 & 5 & 4 & 4 \\
\hline 
0.96 & 
13 & 10 & 8 & 8 & 5 & 6 & 4 & 4 & 4 & 5 & 4 & 4 \\
\hline 
0.97 & 
11 & 10 & 9 & 8 & 5 & 6 & 5 & 4 & 4 & 5 & 4 & 4 \\
\hline 
min & 
11 & 7 & 7 & 8 & 5 & 5 & 4 & 4 & 4 & 5 & 4 & 4 \\
\hline 
prom & 
12.88889 & 10.16667 & 10.11111 & 9.666667 & 5 & 5.111111 & 4.055556 & 4 & 4 & 5 & 4 & 4 \\
\hline 
error & 
\multicolumn{4}{c|}{10.70833333} & 
\multicolumn{4}{c|}{4.541666667} & 
\multicolumn{4}{c|}{4.25}\\ 
\hline 
\end{tabular} 
}
\begin{center}
\vskip 0.2cm
{\small{Fuente: Elaboración propia}}
\end{center}
\end{table}
\end{center}

\vskip -0.5cm

\begin{center}
\begin{table}[H]
\centering
\caption{\small{Resultados del testeo del sistema para BC = 20, 22 y 24 con M = 39.}}
\label{table:tabla3.9}
\vskip 0.2cm
\scalebox{0.54}{
\begin{tabular}{|c|c|c|c|c|c|c|c|c|c|c|c|c|c|}
\hline 
{} & 
\multicolumn{4}{c|}{BC=20 M=39 V=30 P=S1 R=40} & 
\multicolumn{4}{c|}{BC=22 M=39 V=30 P=S1 R=40} & 
\multicolumn{4}{c|}{BC=24 M=39 V=30 P=S1 R=40}  \\ 
\hline 
$\alpha$ & 
S=16 & S=15 & S=14 & S=12 & 
S=16 & S=15 & S=14 & S=12 & 
S=16 & S=15 & S=14 & S=12 \\ 
\hline 
0.8 & 
18 & 17 & 17 & 14 & 11 & 11 & 11 & 5 & 12 & 10 & 9 & 5 \\
\hline 
0.81 & 
17 & 20 & 19 & 15 & 11 & 11 & 11 & 5 & 12 & 10 & 9 & 5 \\
\hline 
0.82 & 
18 & 18 & 17 & 16 & 11 & 11 & 11 & 5 & 12 & 10 & 9 & 5 \\
\hline 
0.83 & 
20 & 16 & 19 & 13 & 11 & 11 & 11 & 5 & 12 & 10 & 9 & 5 \\
\hline 
0.84 & 
20 & 14 & 17 & 13 & 11 & 11 & 11 & 5 & 12 & 10 & 9 & 5 \\
\hline 
0.85 & 
18 & 15 & 18 & 15 & 11 & 11 & 11 & 5 & 12 & 10 & 9 & 5 \\
\hline 
0.86 & 
17 & 14 & 18 & 14 & 11 & 11 & 11 & 5 & 12 & 10 & 9 & 5 \\
\hline 
0.87 & 
14 & 13 & 16 & 12 & 11 & 11 & 11 & 5 & 12 & 10 & 9 & 5 \\
\hline 
0.88 & 
17 & 14 & 16 & 14 & 11 & 11 & 11 & 5 & 12 & 10 & 9 & 5 \\
\hline 
0.89 & 
17 & 15 & 15 & 12 & 11 & 11 & 11 & 5 & 12 & 10 & 9 & 5 \\
\hline 
0.9 & 
16 & 13 & 17 & 12 & 11 & 11 & 11 & 5 & 12 & 10 & 9 & 5 \\
\hline 
0.91 & 
17 & 16 & 17 & 13 & 11 & 11 & 11 & 5 & 12 & 10 & 9 & 5 \\
\hline 
0.92 & 
16 & 16 & 16 & 12 & 11 & 11 & 11 & 5 & 12 & 10 & 9 & 5 \\
\hline 
0.93 & 
16 & 17 & 14 & 13 & 11 & 11 & 11 & 5 & 12 & 10 & 9 & 5 \\
\hline 
0.94 & 
16 & 16 & 17 & 14 & 11 & 11 & 11 & 5 & 12 & 10 & 9 & 5 \\
\hline 
0.95 & 
16 & 14 & 14 & 14 & 11 & 11 & 11 & 5 & 12 & 10 & 9 & 5 \\
\hline 
0.96 & 
16 & 15 & 14 & 14 & 11 & 11 & 11 & 5 & 11 & 10 & 10 & 5 \\
\hline 
0.97 & 
15 & 16 & 14 & 12 & 11 & 11 & 11 & 5 & 11 & 10 & 10 & 5 \\
\hline 
min & 
14 & 13 & 14 & 12 & 11 & 11 & 11 & 5 & 11 & 10 & 9 & 5 \\
\hline 
prom & 
16.88889 & 15.5 & 16.38889 & 12.44444 & 11 & 11 & 11 & 5 & 11.88889 & 10 & 9.111111 & 5 \\
\hline 
error & 
\multicolumn{4}{c|}{15.55555556} & 
\multicolumn{4}{c|}{9.5} & 
\multicolumn{4}{c|}{9}\\ 
\hline 
\end{tabular} 
}
\begin{center}
\vskip 0.2cm
{\small{Fuente: Elaboración propia}}
\end{center}
\end{table}
\end{center}

\vskip -0.5cm
Como podemos ver en la Tabla \ref{table:tabla3.3} para $M = 13$ el mejor resultado obtenido fue con $BC = 22$ donde el error promedio fue de 3.5, en la Tabla \ref{table:tabla3.5} para $M = 15$ el mejor resultado obtenido fue con $BC = 24$ donde el error promedio fue de 3.3, en la Tabla \ref{table:tabla3.7} para $M = 26$ el mejor valor obtenido fue con $BC = 24$ donde el error promedio fue de 4.2 y en la Tabla \ref{table:tabla3.9} para $M = 39$ el mejor resultado obtenido fue con $BC = 24$ donde el error promedio fue de 9.0, podemos concluir de este primer testeo del sistema que se obtuvieron mejores resultados con MFCC ($M = 13, BC = 22$) y ($M = 15, BC = 24$), que los MFCC Delta y Doble Delta.

\subsection{Hallar V}
En esta segunda prueba hallaremos \textit{V}, parámetro usado para la segmentación o ventaneamiento de una señal de voz, teniendo la siguiente configuración del algoritmo:\\
- Filtro Preénfasis \\
\hspace*{1cm} $\alpha$ = Desde 0.80 hasta 0.97 \\
- Segmentación por Hamming \\
\hspace*{1cm} V = 28, 30 y 32 ; \qquad S = 12, 14, 15 y 16 \\
- MFCC \\
\hspace*{1cm} BC = 22 y 24 ; \qquad M = 13 y 15 \\
- DTW \\
\hspace*{1cm} P = 1 Simétrico ; \qquad R = 40 ; \qquad D = Euclidiana Cuadrática
\vskip 0.5cm
En la Tabla \ref{table:tabla3.10} se muestran los mejores resultados para esta prueba, como podemos observar se obtuvieron mejores resultados con $V = 28$, $BC = 24$ y $M = 15$ con un error promedio de 2.9.
\begin{center}
\begin{table}[H]
\centering
\caption{\small{Resultados del testeo del sistema para V = 28.}}
\label{table:tabla3.10}
\vskip 0.2cm
\scalebox{0.54}{
\begin{tabular}{|c|c|c|c|c|c|c|c|c|c|}
\hline 
{} & 
\multicolumn{4}{c|}{BC=22 M=13 V=28 P=S1 R=40} & 
\multicolumn{4}{c|}{BC=24 M=15 V=28 P=S1 R=40} \\ 
\hline 
$\alpha$ & 
S=16 & S=15 & S=14 & S=12 & 
S=16 & S=15 & S=14 & S=12\\ 
\hline 
0.8 & 
4 & 3 & 3 & 3 & 3 & 3 & 3 & 4 \\
\hline 
0.81 & 
4 & 3 & 3 & 3 & 3 & 3 & 3 & 4 \\
\hline 
0.82 & 
4 & 3 & 3 & 3 & 3 & 3 & 3 & 4 \\
\hline 
0.83 & 
4 & 3 & 3 & 3 & 3 & 3 & 3 & 4 \\
\hline 
0.84 & 
4 & 3 & 3 & 3 & 3 & 3 & 3 & 4 \\
\hline 
0.85 & 
4 & 3 & 3 & 3 & 3 & 3 & 3 & 4 \\
\hline 
0.86 & 
4 & 3 & 3 & 3 & 3 & 2 & 3 & 3 \\
\hline 
0.87 & 
4 & 3 & 3 & 3 & 3 & 2 & 3 & 3 \\
\hline 
0.88 & 
3 & 3 & 3 & 3 & 3 & 2 & 3 & 3 \\
\hline 
0.89 & 
3 & 3 & 3 & 3 & 3 & 2 & 3 & 3 \\
\hline 
0.9 & 
3 & 3 & 3 & 3 & 3 & 2 & 3 & 3 \\
\hline 
0.91 & 
3 & 3 & 3 & 3 & 3 & 2 & 3 & 3 \\
\hline 
0.92 & 
3 & 3 & 3 & 3 & 3 & 2 & 3 & 3 \\
\hline 
0.93 & 
3 & 3 & 3 & 3 & 3 & 3 & 3 & 3 \\
\hline 
0.94 & 
3 & 3 & 3 & 3 & 3 & 3 & 3 & 3 \\
\hline 
0.95 & 
3 & 3 & 3 & 3 & 3 & 3 & 3 & 3 \\
\hline 
0.96 & 
3 & 3 & 3 & 3 & 3 & 3 & 3 & 3 \\
\hline 
0.97 & 
3 & 3 & 3 & 3 & 3 & 3 & 3 & 3 \\
\hline 
min & 
3 & 3 & 3 & 3 & 3 & 2 & 3 & 3 \\
\hline 
prom & 
3.4444444 & 3 & 3 & 3 & 3 & 2.61111111 & 3 & 3.3333333 \\
\hline 
error & 
\multicolumn{4}{c|}{3.111111111} & 
\multicolumn{4}{c|}{2.986111111} \\ 
\hline 
\end{tabular} 
}
\begin{center}
\vskip 0.2cm
{\small{Fuente: Elaboración propia}}
\end{center}
\end{table}
\end{center}

\subsection{Hallar P}
En esta tercera prueba hallaremos \textit{P}, parámetro del algoritmo DTW usado para la comparación de patrones característicos de las señales de voz, teniendo la siguiente configuración del algoritmo: \\
- Filtro Preénfasis \\
\hspace*{1cm} $\alpha$ = Desde 0.80 hasta 0.97 \\
- Segmentación por Hamming \\
\hspace*{1cm} V = 28 ; \qquad S = 12, 14, 15 y 16 \\
- MFCC \\
\hspace*{1cm} BC = 24 ; \qquad M = 15 \\
- DTW \\
\hspace*{1cm} P = 0, 1/2, 1 y 2 (Simétrico y Asimétrico) ; \qquad R = 40 ; \qquad D = Euclidiana Cuadrática
\vskip -0.5cm
\begin{center}
\begin{table}[H]
\centering
\caption{\small{Resultados del testeo del sistema para P = 0 simétrico y asimétrico.}}
\label{table:tabla3.13}
\scalebox{0.52}{
\begin{tabular}{|c|c|c|c|c|c|c|c|c|c|}
\hline 
{} & 
\multicolumn{4}{c|}{BC=24 M=15 V=28 P=S0 R=40} & 
\multicolumn{4}{c|}{BC=24 M=15 V=28 P=A0 R=40} \\ 
\hline 
$\alpha$ & 
S=16 & S=15 & S=14 & S=12 & 
S=16 & S=15 & S=14 & S=12 \\ 
\hline 
0.8 & 
15 & 21 & 26 & 42 & 25 & 32 & 34 & 54 \\
\hline 
0.81 & 
15 & 21 & 26 & 42 & 25 & 32 & 34 & 54 \\
\hline 
0.82 & 
15 & 21 & 27 & 42 & 25 & 32 & 34 & 54 \\
\hline 
0.83 & 
15 & 21 & 27 & 42 & 25 & 32 & 34 & 54 \\
\hline 
0.84 & 
15 & 21 & 27 & 42 & 25 & 32 & 34 & 54 \\
\hline 
0.85 & 
15 & 21 & 27 & 42 & 25 & 32 & 34 & 54 \\
\hline 
0.86 & 
15 & 21 & 28 & 42 & 25 & 32 & 34 & 54 \\
\hline 
0.87 & 
15 & 21 & 28 & 43 & 25 & 32 & 34 & 54 \\
\hline 
0.88 & 
15 & 21 & 28 & 43 & 25 & 32 & 34 & 54 \\
\hline 
0.89 & 
15 & 21 & 28 & 43 & 25 & 32 & 34 & 54 \\
\hline 
0.9 & 
15 & 21 & 28 & 43 & 25 & 32 & 34 & 54 \\
\hline 
0.91 & 
15 & 21 & 28 & 43 & 25 & 32 & 34 & 54 \\
\hline 
0.92 & 
15 & 21 & 28 & 44 & 25 & 32 & 34 & 54 \\
\hline 
0.93 & 
16 & 21 & 28 & 44 & 25 & 32 & 34 & 54 \\
\hline 
0.94 & 
16 & 21 & 28 & 44 & 25 & 32 & 35 & 54 \\
\hline 
0.95 & 
16 & 21 & 28 & 44 & 24 & 32 & 35 & 54 \\
\hline 
0.96 & 
16 & 21 & 28 & 45 & 24 & 32 & 35 & 53 \\
\hline 
0.97 & 
16 & 21 & 28 & 45 & 24 & 32 & 36 & 53 \\
\hline 
min & 
15 & 21 & 26 & 42 & 24 & 32 & 34 & 53 \\
\hline 
prom & 
15.277778 & 21.055556 & 27.555556 & 43.055556 & 24.833333 & 32 & 34.277778 & 53.888889 \\
\hline 
error & 
\multicolumn{4}{c|}{26.73611111} & 
\multicolumn{4}{c|}{36.25} \\ 
\hline 
\end{tabular} 
}
\begin{center}
\vskip 0.1cm
{\small{Fuente: Elaboración propia}}
\end{center}
\end{table}
\end{center}
\vskip -1.0cm

\newpage
\begin{center}
\begin{table}[H]
\centering
\caption{\small{Resultados del testeo del sistema para P = 1 simétrico y asimétrico.}}
\label{table:tabla3.14}
\vskip 0.2cm
\scalebox{0.54}{
\begin{tabular}{|c|c|c|c|c|c|c|c|c|c|}
\hline 
{} & 
\multicolumn{4}{c|}{BC=24 M=15 V=28 P=S1 R=40} & 
\multicolumn{4}{c|}{BC=24 M=15 V=28 P=A1 R=40} \\ 
\hline 
$\alpha$ & 
S=16 & S=15 & S=14 & S=12 & 
S=16 & S=15 & S=14 & S=12\\ 
\hline 
0.8 & 
1 & 2 & 2 & 3 & 4 & 6 & 4 & 4 \\
\hline 
0.81 & 
1 & 2 & 2 & 3 & 4 & 5 & 4 & 4 \\
\hline 
0.82 & 
1 & 2 & 2 & 3 & 4 & 5 & 4 & 4 \\
\hline 
0.83 & 
1 & 2 & 2 & 3 & 4 & 5 & 4 & 4 \\
\hline 
0.84 & 
1 & 2 & 2 & 3 & 4 & 5 & 4 & 4 \\
\hline 
0.85 & 
1 & 2 & 2 & 3 & 4 & 5 & 4 & 4 \\
\hline 
0.86 & 
1 & 2 & 2 & 3 & 4 & 5 & 4 & 4 \\
\hline 
0.87 & 
1 & 2 & 2 & 3 & 4 & 5 & 4 & 4 \\
\hline 
0.88 & 
1 & 2 & 2 & 3 & 4 & 5 & 4 & 4 \\
\hline 
0.89 & 
1 & 2 & 2 & 3 & 4 & 5 & 4 & 4 \\
\hline 
0.9 & 
1 & 2 & 2 & 3 & 4 & 5 & 4 & 4 \\
\hline 
0.91 & 
1 & 2 & 2 & 3 & 4 & 5 & 4 & 4 \\
\hline 
0.92 & 
1 & 2 & 2 & 3 & 4 & 5 & 4 & 4 \\
\hline 
0.93 & 
1 & 2 & 2 & 3 & 4 & 5 & 4 & 4 \\
\hline 
0.94 & 
1 & 2 & 2 & 3 & 4 & 5 & 5 & 4 \\
\hline 
0.95 & 
1 & 2 & 2 & 3 & 4 & 5 & 5 & 4 \\
\hline 
0.96 & 
1 & 2 & 2 & 3 & 4 & 5 & 5 & 4 \\
\hline 
0.97 & 
2 & 2 & 2 & 3 & 4 & 5 & 5 & 5 \\
\hline 
min & 
1 & 2 & 2 & 3 & 4 & 5 & 4 & 4 \\
\hline 
prom & 
1.0555556 & 2 & 2 & 3 & 4 & 5.0555556 & 4.2222222 & 4.0555556 \\
\hline 
error & 
\multicolumn{4}{c|}{2.013888889} & 
\multicolumn{4}{c|}{4.333333333} \\ 
\hline 
\end{tabular} 
}
\begin{center}
\vskip 0.2cm
{\small{Fuente: Elaboración propia}}
\end{center}
\end{table}
\end{center}

\begin{center}
\begin{table}[H]
\centering
\caption{\small{Resultados del testeo del sistema para P = 1/2 simétrico y asimétrico.}}
\label{table:tabla3.15}
\vskip 0.2cm
\scalebox{0.54}{
\begin{tabular}{|c|c|c|c|c|c|c|c|c|c|}
\hline 
{} & 
\multicolumn{4}{c|}{BC=24 M=15 V=28 P=S1/2 R=40} & 
\multicolumn{4}{c|}{BC=24 M=15 V=28 P=A1/2 R=40} \\ 
\hline 
$\alpha$ & 
S=16 & S=15 & S=14 & S=12 & 
S=16 & S=15 & S=14 & S=12\\ 
\hline 
0.8 & 
3 & 3 & 3 & 4 & 3 & 3 & 3 & 3 \\
\hline 
0.81 & 
3 & 3 & 3 & 4 & 3 & 3 & 3 & 3 \\
\hline 
0.82 & 
3 & 3 & 3 & 4 & 3 & 3 & 3 & 3 \\
\hline 
0.83 & 
3 & 3 & 3 & 4 & 3 & 3 & 3 & 3 \\
\hline 
0.84 & 
3 & 3 & 3 & 4 & 3 & 3 & 3 & 3 \\
\hline 
0.85 & 
3 & 3 & 3 & 4 & 3 & 3 & 3 & 3 \\
\hline 
0.86 & 
3 & 2 & 3 & 3 & 3 & 3 & 3 & 3 \\
\hline 
0.87 & 
3 & 2 & 3 & 3 & 3 & 3 & 3 & 3 \\
\hline 
0.88 & 
3 & 2 & 3 & 3 & 3 & 3 & 3 & 3 \\
\hline 
0.89 & 
3 & 2 & 3 & 3 & 3 & 3 & 3 & 3 \\
\hline 
0.9 & 
3 & 2 & 3 & 3 & 3 & 3 & 3 & 3 \\
\hline 
0.91 & 
3 & 2 & 3 & 3 & 3 & 3 & 3 & 3 \\
\hline 
0.92 & 
3 & 2 & 3 & 3 & 3 & 3 & 3 & 3 \\
\hline 
0.93 & 
3 & 3 & 3 & 3 & 4 & 3 & 3 & 3 \\
\hline 
0.94 & 
3 & 3 & 3 & 3 & 4 & 3 & 3 & 3 \\
\hline 
0.95 & 
3 & 3 & 3 & 3 & 4 & 3 & 3 & 3 \\
\hline 
0.96 & 
3 & 3 & 3 & 3 & 4 & 3 & 3 & 3 \\
\hline 
0.97 & 
3 & 3 & 3 & 3 & 4 & 3 & 4 & 3 \\
\hline 
min & 
3 & 2 & 3 & 3 & 3 & 3 & 3 & 3 \\
\hline 
prom & 
3 & 2.6111111 & 3 & 3.3333333 & 3.2777778 & 3 & 3.0555556 & 3 \\
\hline 
error & 
\multicolumn{4}{c|}{2.986111111} & 
\multicolumn{4}{c|}{3.083333333} \\ 
\hline 
\end{tabular} 
}
\begin{center}
\vskip 0.2cm
{\small{Fuente: Elaboración propia}}
\end{center}
\end{table}
\end{center}

\begin{center}
\begin{table}[H]
\centering
\caption{\small{Resultados del testeo del sistema para P = 2 simétrico y asimétrico.}}
\label{table:tabla3.16}
\vskip 0.2cm
\scalebox{0.54}{
\begin{tabular}{|c|c|c|c|c|c|c|c|c|c|}
\hline 
{} & 
\multicolumn{4}{c|}{BC=24 M=15 V=28 P=S2 R=40} & 
\multicolumn{4}{c|}{BC=24 M=15 V=28 P=A2 R=40} \\ 
\hline 
$\alpha$ & 
S=16 & S=15 & S=14 & S=12 & 
S=16 & S=15 & S=14 & S=12\\ 
\hline 
0.8 & 
7 & 7 & 5 & 7 & 81 & 79 & 83 & 82 \\
\hline 
0.81 & 
7 & 7 & 5 & 7 & 81 & 79 & 83 & 82 \\
\hline 
0.82 & 
7 & 7 & 5 & 7 & 81 & 79 & 83 & 82 \\
\hline 
0.83 & 
7 & 7 & 5 & 7 & 81 & 79 & 83 & 82 \\
\hline 
0.84 & 
7 & 7 & 5 & 7 & 81 & 79 & 83 & 82 \\
\hline 
0.85 & 
7 & 7 & 5 & 7 & 81 & 79 & 83 & 82 \\
\hline 
0.86 & 
7 & 7 & 5 & 7 & 81 & 79 & 83 & 82 \\
\hline 
0.87 & 
7 & 7 & 5 & 7 & 81 & 80 & 83 & 82 \\
\hline 
0.88 & 
7 & 7 & 5 & 7 & 81 & 80 & 83 & 82 \\
\hline 
0.89 & 
7 & 7 & 5 & 7 & 81 & 80 & 83 & 82 \\
\hline 
0.9 & 
7 & 7 & 5 & 7 & 81 & 80 & 83 & 82 \\
\hline 
0.91 & 
7 & 7 & 5 & 7 & 81 & 80 & 83 & 82 \\
\hline 
0.92 & 
7 & 7 & 5 & 7 & 81 & 80 & 83 & 82 \\
\hline 
0.93 & 
7 & 7 & 5 & 7 & 81 & 80 & 83 & 82 \\
\hline 
0.94 & 
7 & 7 & 5 & 7 & 81 & 80 & 83 & 82 \\
\hline 
0.95 & 
7 & 7 & 5 & 7 & 81 & 79 & 83 & 82 \\
\hline 
0.96 & 
7 & 7 & 5 & 7 & 81 & 79 & 83 & 82 \\
\hline 
0.97 & 
7 & 7 & 5 & 7 & 81 & 79 & 83 & 82 \\
\hline 
min & 
7 & 7 & 5 & 7 & 81 & 79 & 83 & 82 \\
\hline 
prom & 
7 & 7 & 5 & 7 & 81 & 79.444444 & 83 & 82 \\
\hline 
error & 
\multicolumn{4}{c|}{6.5} & 
\multicolumn{4}{c|}{3.81.36111111} \\ 
\hline 
\end{tabular} 
}
\begin{center}
\vskip 0.2cm
{\small{Fuente: Elaboración propia}}
\end{center}
\end{table}
\end{center}

A partir de los resultados obtenidos y vistos en las Tablas \ref{table:tabla3.13}, \ref{table:tabla3.14}, \ref{table:tabla3.15} y \ref{table:tabla3.16} podemos concluir que se obtuvieron mejores resultados para $P = 1$ simétrico con un error promedio de 2.0, por lo que se cumplió con lo que se dijo en la teoría para el valor de $P$, y en segundo lugar para $P = 1/2$ simétrico con un error promedio de 2.9.

\subsection{Hallar S}
En esta cuarta prueba hallaremos \textit{S}, parámetro usado para la segmentación o ventaneamiento de una señal de voz, teniendo la siguiente configuración del algoritmo: \\
- Filtro Preénfasis \\
\hspace*{1cm} $\alpha$ = Desde 0.80 hasta 0.97 \\
- Segmentación por Hamming \\
\hspace*{1cm} V = 28 ; \qquad S = 12, 14, 15, 16, 17, 18, 19, 20 y 21 \\
- MFCC \\
\hspace*{1cm} BC = 24 ; \qquad M = 15 \\
- DTW \\
\hspace*{1cm} P = 1/2 y 1 Simétrico ; \qquad R = 40 ; \qquad D = Euclidiana Cuadrática
\vskip -1.0cm
\begin{center}
\begin{table}[H]
\centering
\caption{\small{Resultados del testeo del sistema para P = 1/2 simétrico.}}
\label{table:tabla3.17}
\vskip 0.2cm
\scalebox{0.54}{
\begin{tabular}{|c|c|c|c|c|c|c|c|c|c|c|}
\hline 
{} & 
\multicolumn{9}{c|}{BC=24 M=15 V=28 P=S1/2 R=40} \\ 
\hline 
$\alpha$ & 
S=21 & S=20 & S=19 & S=18 & 
S=17 & S=16 & S=15 & S=14 & S=12\\ 
\hline 
0.8 & 
1 & 2 & 1 & 1 & 1 & 1 & 2 & 2 & 3 \\
\hline 
0.81 & 
1 & 2 & 1 & 1 & 1 & 1 & 2 & 2 & 3 \\
\hline 
0.82 & 
1 & 2 & 1 & 1 & 1 & 1 & 2 & 2 & 3 \\
\hline 
0.83 & 
1 & 2 & 1 & 1 & 1 & 1 & 2 & 2 & 3 \\
\hline 
0.84 & 
1 & 2 & 1 & 1 & 1 & 1 & 2 & 2 & 3 \\
\hline 
0.85 & 
1 & 2 & 1 & 1 & 1 & 1 & 2 & 2 & 3 \\
\hline 
0.86 & 
1 & 2 & 1 & 1 & 1 & 1 & 2 & 2 & 3 \\
\hline 
0.87 & 
1 & 2 & 1 & 1 & 1 & 1 & 2 & 2 & 3 \\
\hline 
0.88 & 
1 & 2 & 1 & 1 & 1 & 1 & 2 & 2 & 3 \\
\hline 
0.89 & 
1 & 2 & 1 & 1 & 1 & 1 & 2 & 2 & 3 \\
\hline 
0.9 & 
1 & 2 & 1 & 1 & 1 & 1 & 2 & 2 & 3 \\
\hline 
0.91 & 
1 & 2 & 1 & 1 & 1 & 1 & 2 & 2 & 3 \\
\hline 
0.92 & 
1 & 2 & 1 & 1 & 1 & 1 & 2 & 2 & 3 \\
\hline 
0.93 & 
1 & 2 & 1 & 1 & 1 & 1 & 2 & 2 & 3 \\
\hline 
0.94 & 
1 & 2 & 1 & 1 & 1 & 1 & 2 & 2 & 3 \\
\hline 
0.95 & 
1 & 2 & 1 & 1 & 1 & 1 & 2 & 2 & 3 \\
\hline 
0.96 & 
2 & 2 & 2 & 1 & 1 & 1 & 2 & 2 & 3 \\
\hline 
0.97 & 
2 & 2 & 2 & 1 & 1 & 2 & 2 & 2 & 3 \\
\hline 
min & 
1 & 2 & 1 & 1 & 1 & 1 & 2 & 2 & 3 \\
\hline 
prom & 
1.1111111 & 2 & 1.1111111 & 1 & 1 & 1.0555556 & 2 & 2 & 3 \\
\hline 
error & 
\multicolumn{9}{c|}{1.586419753} \\ 
\hline 
\end{tabular} 
}
\begin{center}
\vskip 0.2cm
{\small{Fuente: Elaboración propia}}
\end{center}
\end{table}
\end{center}

\begin{center}
\begin{table}[H]
\centering
\caption{\small{Resultados del testeo del sistema para P = 1 simétrico.}}
\label{table:tabla3.18}
\vskip 0.2cm
\scalebox{0.54}{
\begin{tabular}{|c|c|c|c|c|c|c|c|c|c|c|}
\hline 
{} & 
\multicolumn{9}{c|}{BC=24 M=15 V=28 P=S1 R=40} \\ 
\hline
$\alpha$ & 
S=21 & S=20 & S=19 & S=18 & 
S=17 & S=16 & S=15 & S=14 & S=12\\ 
\hline 
0.8 & 
3 & 2 & 2 & 4 & 2 & 1 & 1 & 3 & 3 \\
\hline 
0.81 & 
3 & 2 & 2 & 4 & 2 & 1 & 1 & 3 & 3 \\
\hline 
0.82 & 
3 & 2 & 2 & 4 & 2 & 1 & 1 & 3 & 3 \\
\hline 
0.83 & 
3 & 2 & 2 & 4 & 2 & 1 & 1 & 3 & 3 \\
\hline 
0.84 & 
3 & 2 & 2 & 4 & 2 & 1 & 1 & 3 & 3 \\
\hline 
0.85 & 
3 & 2 & 2 & 4 & 2 & 1 & 1 & 3 & 3 \\
\hline 
0.86 & 
3 & 2 & 2 & 4 & 2 & 1 & 1 & 3 & 3 \\
\hline 
0.87 & 
3 & 2 & 2 & 4 & 2 & 1 & 1 & 3 & 3 \\
\hline 
0.88 & 
3 & 2 & 2 & 4 & 2 & 1 & 1 & 3 & 3 \\
\hline 
0.89 & 
3 & 2 & 2 & 4 & 2 & 1 & 1 & 3 & 3 \\
\hline 
0.9 & 
3 & 2 & 2 & 4 & 2 & 1 & 1 & 3 & 3 \\
\hline 
0.91 & 
3 & 2 & 2 & 4 & 2 & 1 & 1 & 3 & 3 \\
\hline 
0.92 & 
3 & 2 & 2 & 4 & 2 & 1 & 1 & 3 & 3 \\
\hline 
0.93 & 
3 & 2 & 2 & 4 & 2 & 1 & 1 & 3 & 3 \\
\hline 
0.94 & 
3 & 2 & 2 & 4 & 2 & 1 & 1 & 3 & 3 \\
\hline 
0.95 & 
3 & 2 & 2 & 4 & 2 & 1 & 1 & 3 & 4 \\
\hline 
0.96 & 
3 & 2 & 2 & 4 & 2 & 1 & 1 & 3 & 4 \\
\hline 
0.97 & 
3 & 2 & 2 & 3 & 2 & 1 & 1 & 3 & 3 \\
\hline 
min & 
3 & 2 & 2 & 3 & 2 & 1 & 1 & 3 & 3 \\
\hline 
prom & 
3 & 2 & 2 & 3.9444444 & 2 & 1 & 1 & 3 & 3.1111111 \\
\hline 
error & 
\multicolumn{9}{c|}{2.339506173} \\ 
\hline 
\end{tabular} 
}
\begin{center}
\vskip 0.2cm
{\small{Fuente: Elaboración propia}}
\end{center}
\end{table}
\end{center}

A partir de los resultados obtenidos y vistos en las Tablas \ref{table:tabla3.17} y \ref{table:tabla3.18} podemos concluir que se obtuvieron mejores resultados para $P = 1/2$ Simétrico con un error promedio de 1.58 donde $S = 17$ y $S = 18$ obtuvieron un $prom = 1$.
\vskip 0.5cm
Ahora pasaremos a decidir el valor de $S = 17$ o $S = 18$, ya que ambos tienen de error promedio 1.0, para ello realizaremos un calculo de la diferencia de ese error, estos resultados se muestran en la Tabla \ref{table:tabla3.19}.
\vskip 0.5cm
\begin{center}
\begin{table}[H]
\centering
\caption{\small{Resultados del testeo del sistema para P = 1 simétrico con S = 17 y 18.}}
\label{table:tabla3.19}
\scalebox{0.53}{
\begin{tabular}{|c|c|c|c|}
\hline 
{} & 
\multicolumn{2}{c|}{BC=24 M=15 V=28 P=S1 R=40} \\ 
\hline 
$\alpha$ & 
S=17 & S=18 \\ 
\hline 
0.8 & 
0.1094124943 & 0.177730377 \\
\hline 
0.81 & 
0.109218433 & 0.1777551714 \\
\hline 
0.82 & 
0.109042847 & 0.177391216 \\
\hline 
0.83 & 
0.108885997 & 0.177253609 \\
\hline 
0.84 & 
0.108749186 & 0.177144889 \\
\hline 
0.85 & 
0.108632233 & 0.177072825 \\
\hline 
0.86 & 
0.108533699 & 0.177047698 \\
\hline 
0.87 & 
0.1084665 & 0.177083403 \\
\hline 
0.88 & 
0.108441072 & 0.177199142 \\
\hline 
0.89 & 
0.108469676 & 0.177422935 \\
\hline 
0.9 & 
0.108568203 & 0.177796984 \\
\hline 
0.91 & 
0.108757903 & 0.178388854 \\
\hline 
0.92 & 
0.109073457 & 0.179303729 \\
\hline 
0.93 & 
0.109559355 & 0.180665003 \\
\hline 
0.94 & 
0.110277193 & 0.182589138 \\
\hline 
0.95 & 
0.111342347 & 0.185213138 \\
\hline 
0.96 & 
0.112971013 & 0.18867958 \\
\hline 
0.97 & 
0.115673769 & 0.193678132 \\
\hline 
min & 
0.108441072 & 0.177047698 \\
\hline 
prom & 
0.109412494 & 0.179956242 \\
\hline 
error & 1 & 1 \\ 
\hline 
\end{tabular} 
}
\begin{center}
\vskip 0.1cm
{\small{Fuente: Elaboración propia}}
\end{center}
\end{table}
\end{center}
\vskip -0.7cm
Como podemos ver en la Tabla \ref{table:tabla3.19} se obtuvo mejores resultados con $S = 17$ con una diferencia de error promedio de $prom = 0.10$.

\subsection{Hallar r}
En esta quinta prueba hallaremos \textit{r}, parámetro usado por el algoritmo DTW empleado para la comparación de patrones de las señales de voz, teniendo la siguiente configuración del algoritmo:\\
- Filtro Preénfasis \\
\hspace*{1cm} $\alpha$ = Desde 0.80 hasta 0.97 \\
- Segmentación por Hamming \\
\hspace*{1cm} V = 28 ; \qquad S = 17 \\
- MFCC \\
\hspace*{1cm} BC = 24 ; \qquad M = 15 \\
- DTW \\
\hspace*{1cm} P = 1/2 y 1 Simétrico ; R = 20, 25, 30, 35, 40, 45, 50, 55 y 60 ; D = Euclidiana Cuadrática
\vskip -1.0cm
\begin{center}
\begin{table}[H]
\centering
\caption{\small{Resultados del testeo del sistema para P = 1/2 simétrico con S = 17 y 18.}}
\label{table:tabla3.20}
\scalebox{0.53}{
\begin{tabular}{|c|c|c|c|c|c|c|c|c|c|c|}
\hline 
{} & 
\multicolumn{9}{c|}{BC=24 M=15 V=28 S=17 P=S1/2} \\ 
\hline
$\alpha$ & 
R=20 & R=15 & R=30 & R=35 & R=40 & R=45 & R=50 & R=55 & R=60 \\ 
\hline 
0.8 & 
14 & 5 & 2 & 2 & 1 & 1 & 1 & 1 & 1 \\
\hline 
0.81 & 
14 & 5 & 2 & 2 & 1 & 1 & 1 & 1 & 1 \\
\hline 
0.82 & 
14 & 5 & 2 & 2 & 1 & 1 & 1 & 1 & 1 \\
\hline 
0.83 & 
13 & 5 & 2 & 2 & 1 & 1 & 1 & 1 & 1 \\
\hline 
0.84 & 
13 & 5 & 2 & 2 & 1 & 1 & 1 & 1 & 1 \\
\hline 
0.85 & 
13 & 5 & 2 & 2 & 1 & 1 & 1 & 1 & 1 \\
\hline 
0.86 & 
13 & 5 & 2 & 2 & 1 & 1 & 1 & 1 & 1 \\
\hline 
0.87 & 
13 & 5 & 2 & 2 & 1 & 1 & 1 & 1 & 1 \\
\hline 
0.88 & 
13 & 5 & 2 & 2 & 1 & 1 & 1 & 1 & 1 \\
\hline 
0.89 & 
13 & 5 & 2 & 2 & 1 & 1 & 1 & 1 & 1 \\
\hline 
0.9 & 
13 & 5 & 2 & 2 & 1 & 1 & 1 & 1 & 1 \\
\hline 
0.91 & 
13 & 5 & 2 & 2 & 1 & 1 & 1 & 1 & 1 \\
\hline 
0.92 & 
13 & 5 & 2 & 2 & 1 & 1 & 1 & 1 & 1 \\
\hline 
0.93 & 
13 & 5 & 2 & 2 & 1 & 1 & 1 & 1 & 1 \\
\hline 
0.94 & 
14 & 5 & 2 & 2 & 1 & 1 & 1 & 1 & 1 \\
\hline 
0.95 & 
14 & 5 & 2 & 2 & 1 & 1 & 1 & 1 & 1 \\
\hline 
0.96 & 
14 & 5 & 2 & 2 & 1 & 1 & 1 & 1 & 1 \\
\hline 
0.97 & 
14 & 5 & 2 & 2 & 1 & 1 & 1 & 1 & 1 \\
\hline 
min & 
13 & 5 & 2 & 2 & 1 & 1 & 1 & 1 & 1 \\
\hline 
prom & 
13.444444 & 5 & 2 & 2 & 1 & 1 & 1 & 1 & 1 \\
\hline 
error & 
\multicolumn{9}{c|}{3.049382716} \\ 
\hline 
\end{tabular} 
}
\begin{center}
\vskip 0.1cm
{\small{Fuente: Elaboración propia}}
\end{center}
\end{table}
\end{center}
\vskip -1.0cm
\begin{center}
\begin{table}[H]
\centering
\caption{\small{Resultados del testeo del sistema para P = 1 simétrico con S = 17 y 18.}}
\label{table:tabla3.21}
\scalebox{0.53}{
\begin{tabular}{|c|c|c|c|c|c|c|c|c|c|c|}
\hline 
{} & 
\multicolumn{9}{c|}{BC=24 M=15 V=28 S=17 P=S1} \\ 
\hline
$\alpha$ & 
R=20 & R=15 & R=30 & R=35 & R=40 & R=45 & R=50 & R=55 & R=60 \\ 
\hline 
0.8 & 
20 & 7 & 3 & 2 & 1 & 1 & 1 & 1 & 1 \\
\hline 
0.81 & 
20 & 7 & 3 & 2 & 1 & 1 & 1 & 1 & 1 \\
\hline 
0.82 & 
20 & 7 & 3 & 2 & 1 & 1 & 1 & 1 & 1 \\
\hline 
0.83 & 
20 & 7 & 3 & 2 & 1 & 1 & 1 & 1 & 1 \\
\hline 
0.84 & 
20 & 7 & 3 & 2 & 1 & 1 & 1 & 1 & 1 \\
\hline 
0.85 & 
20 & 7 & 3 & 2 & 1 & 1 & 1 & 1 & 1 \\
\hline 
0.86 & 
20 & 7 & 3 & 2 & 1 & 1 & 1 & 1 & 1 \\
\hline 
0.87 & 
20 & 7 & 3 & 2 & 1 & 1 & 1 & 1 & 1 \\
\hline 
0.88 & 
20 & 7 & 3 & 2 & 1 & 1 & 1 & 1 & 1 \\
\hline 
0.89 & 
20 & 7 & 3 & 2 & 1 & 1 & 1 & 1 & 1 \\
\hline 
0.9 & 
20 & 7 & 3 & 2 & 1 & 1 & 1 & 1 & 1 \\
\hline 
0.91 & 
20 & 7 & 3 & 2 & 1 & 1 & 1 & 1 & 1 \\
\hline 
0.92 & 
20 & 7 & 3 & 2 & 1 & 1 & 1 & 1 & 1 \\
\hline 
0.93 & 
20 & 7 & 3 & 2 & 1 & 1 & 1 & 1 & 1 \\
\hline 
0.94 & 
20 & 7 & 3 & 2 & 1 & 1 & 1 & 1 & 1 \\
\hline 
0.95 & 
20 & 7 & 3 & 2 & 1 & 1 & 1 & 1 & 1 \\
\hline 
0.96 & 
20 & 7 & 3 & 2 & 1 & 1 & 1 & 1 & 1 \\
\hline 
0.97 & 
20 & 8 & 3 & 2 & 1 & 1 & 1 & 1 & 1 \\
\hline 
min & 
20 & 7 & 3 & 2 & 1 & 1 & 1 & 1 & 1 \\
\hline 
prom & 
20 & 7.0555556 & 3 & 2 & 1 & 1 & 1 & 1 & 1 \\
\hline 
error & 
\multicolumn{9}{c|}{4.117283951} \\ 
\hline 
\end{tabular} 
}
\begin{center}
\vskip 0.1cm
{\small{Fuente: Elaboración propia}}
\end{center}
\end{table}
\end{center}

Como podemos ver en la Tabla \ref{table:tabla3.20} y \ref{table:tabla3.21} con $P = 1/2$ se obtuvieron mejores resultados con $R > 40$.

\subsection{Hallar T}
En esta sexta prueba hallaremos \textit{T}, parámetro usado por el algoritmo de Eliminación de Segmentos Inútiles por Energía empleado para la etapa de Detección de Inicio y Fin de la Señal de Voz, teniendo la siguiente configuración del algoritmo: \\
- Detección de Inicio y Fin de la Señal de Voz (por función de energía) \\
\hspace*{1cm} T = 80, 96, 112, 128, 144, 160, 170, 176, 192, 200, 208, 210, 220, 224, 230, 240, 256, 260, 270 y 272 \\
\hspace*{1cm} E = 1 y 3 número de tramas para el cálculo de los puntos de inicio y fin de la señal de voz \\
- Filtro Preénfasis \\
\hspace*{1cm} $\alpha$ = 0.95 \\
- Segmentación por Hamming \\
\hspace*{1cm} V = 28 ; \qquad S = 17 \\
- MFCC \\
\hspace*{1cm} BC = 24 ; \qquad M = 15 \\
- DTW \\
\hspace*{1cm} P = 1/2 Simétrico ; \qquad R = 40 ; \qquad D = Euclidiana Cuadrática 

\begin{center}
\begin{table}[H]
\centering
\caption{\small{Resultados del testeo del sistema para E = 1 y 3.}}
\label{table:tabla3.22}
\vskip 0.1cm
\scalebox{0.54}{
\begin{tabular}{|c|c|c|c|}
\hline 
{} & 
\multicolumn{2}{c|}{A=0.95 BC=24 M=15 V=28 P=S1 R=40} \\ 
\hline 
T & 
E=1 & E=3 \\ 
\hline 
272 & 
0.1423361234 & 0.131170041 \\
\hline 
270 & 
0.142265312 & 0.169930906 \\
\hline 
260 & 
0.14654467 & 0.138318809 \\
\hline 
256 & 
0.206047218 & 0.089632901 \\
\hline 
240 & 
0.180985061 & 0.099563841 \\
\hline 
230 & 
0.13636777 & 0.123425669 \\
\hline 
224 & 
0.162440198 & 0.122705063 \\
\hline 
0.87 & 
0.1084665 & 0.177083403 \\
\hline 
220 & 
0.152063116 & 0.143037404 \\
\hline 
210 & 
0.130882351 & 0.138373347 \\
\hline 
208 & 
0.14477218 & 0.129921146 \\
\hline 
200 & 
0.163915064 & 0.142663281 \\
\hline 
192 & 
0.136109187 & 0.12011364 \\
\hline 
176 & 
0.177405319 & 0.163441463 \\
\hline 
160 & 
0.183944807 & 0.161748511 \\
\hline 
144 & 
0.182718486 & 0.196339446 \\
\hline 
128 & 
0.206047218 & 0.175164863 \\
\hline 
112 & 
0.196310078 & 0.143973071 \\
\hline 
96 & 
0.182718486 & 0.182723452 \\
\hline 
80 & 
0.180985061 & 0.177249872 \\
\hline 
min & 
0.1308823509 & 0.89632901 \\
\hline 
prom & 
0.1660451424 & 0.177796984 \\
\hline 
error & 1 & 1 \\ 
\hline 
\end{tabular} 
}
\begin{center}
\vskip 0.2cm
{\small{Fuente: Elaboración propia}}
\end{center}
\end{table}
\end{center}

Como podemos ver en la Tabla \ref{table:tabla3.22} con $E = 3$ se obtuvo mejores resultados con un error promedio de $prom = 0.14$ teniendo como error mínimo para $T = 256$, el segundo lugar para $T = 240$ y en tercer lugar para $T = 192$, pero vemos que el valor de error para $T = 192$ tanto en $E = 1$ como para $E = 3$ ocupa el tercer lugar siendo este el más estable por lo que se escogerá a este como mejor valor para la longitud de trama para la eliminación de segmentos inútiles por energía.

\subsection{Hallar V y S}
En esta septima prueba hallaremos \textit{V} y \textit{S}, parámetros usados para la etapa de segmentación o ventaneamiento de la señal de voz, teniendo la siguiente configuración del algoritmo:\\
- Detección de Inicio y Fin de la Señal de Voz (por función de energía) \\
\hspace*{1cm} T = 192 ; \qquad E = 3 \\
- Filtro Preénfasis \\
\hspace*{1cm} $\alpha$ = Desde 0.80 hasta 0.97 \\
- Segmentación por Hamming \\
\hspace*{1cm} V = Desde 20 hasta 31 ; \qquad S = Desde 9 hasta 21 \\
- MFCC \\
\hspace*{1cm} BC = 24 ; \qquad M = 15 \\
- DTW \\
\hspace*{1cm} P = 1/2 Simétrico ; \qquad R = 40 ; \qquad D = Euclidiana Cuadrática 

\begin{center}
\begin{table}[H]
\centering
\caption{\small{Resultados del testeo del sistema para V = 28 y 29.}}
\label{table:tabla3.27}
\vskip 0.2cm
\scalebox{0.54}{
\begin{tabular}{|c|c|c|c|c|c|c|c|c|c|c|c|c|c|c|c|c|c|c|c|}
\hline 
{} & 
\multicolumn{6}{c|}{BC=24 M=15 V=28 T=192 P=S1/2 R=40} &
\multicolumn{6}{c|}{BC=24 M=15 V=29 T=192 P=S1/2 R=40} \\ 
\hline 
$\alpha$ & 
S=19 & S=18 & S=17 & S=16 & S=15 & S=14 & 
S=19 & S=18 & S=17 & S=16 & S=15 & S=14\\ 
\hline 
0.8 & 
1 & 1 & 1 & 1 & 2 & 2 & 1 & 1 & 1 & 1 & 2 & 2  \\
\hline 
0.81 & 
1 & 1 & 1 & 1 & 2 & 2 & 1 & 1 & 1 & 1 & 2 & 2 \\
\hline 
0.82 & 
1 & 1 & 1 & 1 & 2 & 2 & 1 & 1 & 1 & 1 & 2 & 2 \\
\hline 
0.83 & 
1 & 1 & 1 & 1 & 2 & 2 & 1 & 1 & 1 & 1 & 2 & 2 \\
\hline 
0.84 & 
1 & 1 & 1 & 1 & 2 & 2 & 1 & 1 & 1 & 1 & 2 & 2 \\
\hline 
0.85 & 
1 & 1 & 1 & 1 & 2 & 2 & 1 & 1 & 1 & 1 & 2 & 2 \\
\hline 
0.86 & 
1 & 1 & 1 & 1 & 2 & 2 & 1 & 1 & 1 & 1 & 2 & 2 \\
\hline 
0.87 & 
1 & 1 & 1 & 1 & 2 & 2 & 1 & 1 & 1 & 1 & 2 & 2 \\
\hline 
0.88 & 
1 & 1 & 1 & 1 & 2 & 2 & 1 & 1 & 1 & 1 & 2 & 2 \\
\hline 
0.89 & 
1 & 1 & 1 & 1 & 2 & 2 & 1 & 1 & 1 & 1 & 2 & 2 \\
\hline 
0.9 & 
1 & 1 & 1 & 1 & 2 & 2 & 1 & 1 & 1 & 1 & 2 & 2 \\
\hline 
0.91 & 
1 & 1 & 1 & 1 & 2 & 2 & 1 & 1 & 1 & 1 & 2 & 2 \\
\hline 
0.92 & 
1 & 1 & 1 & 1 & 2 & 2 & 1 & 1 & 1 & 1 & 2 & 2 \\
\hline 
0.93 & 
1 & 1 & 1 & 1 & 2 & 2 & 1 & 1 & 1 & 1 & 2 & 2 \\
\hline 
0.94 & 
1 & 1 & 1 & 1 & 2 & 2 & 1 & 1 & 1 & 1 & 2 & 2 \\
\hline 
0.95 & 
1 & 1 & 1 & 1 & 2 & 2 & 1 & 1 & 1 & 1 & 2 & 2 \\
\hline 
0.96 & 
1 & 1 & 1 & 1 & 2 & 2 & 1 & 2 & 2 & 1 & 2 & 2 \\
\hline 
0.97 & 
1 & 1 & 1 & 1 & 2 & 2 & 2 & 2 & 2 & 1 & 2 & 2 \\
\hline 
min & 
1 & 1 & 1 & 1 & 2 & 2 & 1 & 1 & 1 & 1 & 2 & 2 \\
\hline 
prom & 
1.0555556 & 1 & 1 & 1 & 2 & 2 & 1.0555556 & 1.1111111 & 1.1111111 & 1 & 2 & 2 \\
\hline 
error & 
\multicolumn{6}{c|}{1.342592593} & 
\multicolumn{6}{c|}{1.37962963} \\ 
\hline 
\end{tabular} 
}
\begin{center}
\vskip 0.2cm
{\small{Fuente: Elaboración propia}}
\end{center}
\end{table}
\end{center}

Los mejores resultados que se obtuvieron para esta prueba se muestran en la Tabla \ref{table:tabla3.27}, como podemos ver con $V = 28$ se obtuvo el mejor resultado con un error promedio de $error = 1.34$.

\subsection{Hallar d}
En esta octava prueba hallaremos \textit{d}, el tipo de medida de distancia entre dos patrones de señales de voz, usado por el algoritmo DTW en la etapa de \textit{Reconocimiento de Voz}, teniendo la siguiente configuración del algoritmo:\\
- Detección de Inicio y Fin de la Señal de Voz (por función de energía) \\
\hspace*{1cm} T = 192 ; \qquad E = 3 \\
- Filtro Preénfasis \\
\hspace*{1cm} $\alpha$ = Desde 0.80 hasta 0.97 \\
- Segmentación por Hamming \\
\hspace*{1cm} V = 28 ; \qquad S = 18 \\
- MFCC \\
\hspace*{1cm} BC = 24 ; \qquad M = 15 \\
- DTW \\
\hspace*{1cm} P = 1/2 Simétrico ;  \qquad R = 40 ; \qquad D = Euclidiana Cuadrática y Error Cuadrático Medio

\begin{center}
\begin{table}[H]
\centering
\caption{\small{Resultados del testeo del sistema para distancia euclidiana cuadrática y error cuadrático medio.}}
\label{table:tabla3.36}
\vskip 0.2cm
\scalebox{0.54}{
\begin{tabular}{|c|c|c|c|}
\hline 
{} & 
\multicolumn{2}{c|}{BC=24 M=15 V=28 P=S18 T=192 P=S1/2 R=40} \\ 
\hline 
$\alpha$ & 
Euclidiana Cuadrática & Error Cuadrático Medio \\ 
\hline 
0.8 & 
0.335508635 & 0.022367242 \\
\hline 
0.81 & 
0,328620417 & 0.021908028 \\
\hline 
0.82 & 
0.321050444 & 0.021403363 \\
\hline 
0.83 & 
0.312792778 & 0.020852852 \\
\hline 
0.84 & 
0.303845430 & 0.020256362 \\
\hline 
0.85 & 
0.294211547 & 0.019614103 \\
\hline 
0.86 & 
0.283902117 & 0.018926808 \\
\hline 
0.87 & 
0.272940625 & 0.018196042 \\
\hline 
0.88 & 
0.261370472 & 0.017424698 \\
\hline 
0.89 & 
0.249266546 & 0.016617770 \\
\hline 
0.9 & 
0.236753339 & 0.015783556 \\
\hline 
0.91 & 
0.224033462 & 0.014935564 \\
\hline 
0.92 & 
0.211432457 & 0.014095497 \\
\hline 
0.93 & 
0.199467598 & 0.01329784 \\
\hline 
0.94 & 
0.188945966 & 0.012596398 \\
\hline 
0.95 & 
0.181075980 & 0.012071732 \\
\hline 
0.96 & 
0.177493532 & 0.011832902 \\
\hline 
0.97 & 
0.179872047 & 0.011991470 \\
\hline 
min & 
0.177493532 & 0.011832902 \\
\hline 
prom & 
0.253476855 & 0.016898457 \\
\hline 
error & 1 & 1 \\ 
\hline 
\end{tabular} 
}
\begin{center}
\vskip 0.2cm
{\small{Fuente: Elaboración propia}}
\end{center}
\end{table}
\end{center}

\vskip -0.5cm
Como podemos ver en la Tabla \ref{table:tabla3.36} con la Distancia de Error Cuadrático Medio se obtuvo mejores resultados con un error promedio de $prom = 0.016$ teniendo un error mínimo $min=0.011$ con $\alpha =0.96$, además vemos que conforme este valor va aumentando se obtiene mejores resultados, sin embargo vemos que cuando $\alpha = 0.97$ aumenta el error por lo que se escogerá $\alpha =0.95$, valor propuesto en la teoría en la Ecuación \eqref{eq:ecuacion33}.
